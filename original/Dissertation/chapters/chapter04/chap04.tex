%
% File: chap01.tex
% Author: Victor F. Brena-Medina
% Description: Introduction chapter where the biology goes.
%
\let\textcircled=\pgftextcircled
\chapter{Discussion}
\label{chap:intro}

\initial{I}t is known that the United Kingdom's Nuclear industry has a "Safety Culture" wherein materials are designed to experience loads well below their elastic limit \cite{internationalnuclearsafetyadvisorygroup_1991_safety}. In fact, recent nuclear materials have been designed to mitigate fracture even when cracks and other defects are present in the material. \cite{nichols_1981_fracture} For this reason it can be difficult to argue for a change in the way the industry carries out statistical inference if the improvement resulting from said change is marginal, as is often the case with Bayesian statistics. \cite{smidd_2020_bayesian}\\

Conversely, relatively small variations in residual stress can lead to variations in long-term component lifetime of up to decades \cite{webster_2001_residual}, and accurate stress analysis could prove pivotal in determining the lifetime of components which are not regularly replaced, such as pressurisers or steam generators of Pressurised Water Reactors \cite{internationalatomicenergyassociation_2008_heavy}. For this reason, it is worth fully examining potential improvements to current stress analysis methods, and leaving the decision of what constitutes a significant improvement up to the researcher in question.\\

In this chapter, Bayesian linear regression will be evaluated relative to the Nuclear industry standard, Least-Squares regression, in the context of the tests documented in sections 3 and 4 and their implications for the wider Nuclear industry. The limitations of this work will be discussed, along with a justification of the methods used and potential explanations for sources of inconsistency. 



%=======
\section{Comparison of Bayesian Regression with Least-Squares Regression}
\label{sec:sec01}

\subsection{Is the Bayesian Fit Outcome more Accurate?}
\label{subsec:subsec01}

When the standard deviation of errors is known, it is common to use a Normal distribution as a likelihood function for linear regression. For additive errors, a Student's T distribution can be used to account for the additional uncertainty. \cite{zhu_2018_bayesian}\\

If there is little or no uncertainty arising from grain sampling statistics, then the uncertainty associated with diffraction peak position, which is easily resolved by peak width analysis \cite{zhu_2018_bayesian}, is representative of the uncertainty in the measurement. It follows that in this situation, it would be appropriate to use a Normal distribution to describe the uncertainty and would yield the most accurate fit possible.\\ 

Recalling that an improvement in fit quality was only observed for measurements with additive errors, and importantly not for the measurements with only T-distributed errors, this result did not appear to make sense at first; 'Why did a Student T likelihood function not result in a more accurate fit, when the error distribution is, in fact, drawn from a student T distribution?' It is possible that this result is due to the error being 'known' and thus the Normal likelihood function being sufficient for an accurate strain fit. The error was 'known' due to a mistake in the way the Bayesian regression was coded in this case (see section 3.1) which was not apparent until the results were analysed; Specifically, "sigma = bay\_fit ['strain unc']" tells the PyMC3 algorithm that the error associated with each measurement equals the standard deviation of the noise distribution from which the error actually drawn. \\

In the case of measurements with additive errors, this line of code implies that the error which should be associated with each measurement equals that associated with peak position uncertainty only. This was inaccurate enough for the true uncertainty to be unknown to the PyMC3 algorithm , because additional Student's T noise was added to the sample, and thus the T likelihood function yielded more accurate fit outcomes for these measurements because it accounted for the additional uncertainty. \\

Therefore, for measurements with additive errors, the Student's T likelihood function successfully increased the robustness of the fit, with clear improvements in the fit outcomes relative to those generated by Least-Squares Regression (see section 4.4). However, it is unclear from the results of this project whether robust Bayesian regression would produce a more accurate fit for measurements with only one type of error. For now, Bayesian statistics have proven to be useful in the stress analysis of measurements with extreme outliers arising from poor grain sampling, resulting in more accurate fitted stress tensor components than what was obtained using Least-Squares regression.\\ 


\subsection{Does Bayesian Regression Yield More Accurate or Representative Uncertainties?}
\label{subsec:subsec01}

The distribution of Least-Squares fit outcomes was visualised in this report using a simple Bayesian implementation in PyMC3, having established that the two methods are identical for practical purposes (see section 4.1); In practice, uncertainties obtained from Least-Squares regression are in the form of a mean and standard distribution to describe the sought parameter. \\

In contrast, Bayesian regression returns a distribution of potential values for the parameter, which is not necessarily Normal. In the case of samples with additive errors, which were used to simulate the effects of grain sampling statistics (see section 2.2), this was valuable as it allowed for the resulting bi-modal fitted strain distributions. The bi-modal distributions (figures 4.7 and 4.8) enabled a more accurate description of the uncertainty associated with the measurements, which incorporated the potential for the outliers to be accurate.  \\

In short, Bayesian regression offers the ability to easily visualise the distribution of fit outcomes, which is more representative than providing only a standard deviation, particularly in cases where the uncertainty differs throughout the sample. Furthermore, when the error distribution of measurements, and therefore the fit outcome, is not Normally distributed, Bayesian regression is not necessarily restricted by the assumption that it is.\\


\subsection{Is Bayesian Practical for Widespread Industry Use?}
\label{subsec:subsec01}

Ease of implementation is a factor which should be considered when discussing the industry viability of a technique. Of course, relative to Least-Squares regression, Bayesian inference is less easily accessible. As of 2020, its use requires the installation of several packages, already barrier to accessibility in workplace computers which may require admin privileges for downloads, as well as a powerful processor \cite{zhao_2020_fast}. In contrast, Least-Squares can be accessed using Microsoft Excel and most online plotting software. \\

Furthermore, as established in section 4.1, 'thoughtless' \cite{smidd_2020_bayesian} Bayesian regression will yield identical results to Least-Squares regression. The researcher must have an understanding of the nature and origin of uncertainty in their sample and select an appropriate likelihood function and prior strength in order to maximise their chances of obtaining an accurate fit.\\

However, it is worth mentioning that implementing robust Bayesian regression can be less challenging than manually optimising poorly sampled diffraction data, which can take hours \cite{he_2018_twodimensional}, whereas PyMC3 mostly requires initial setup. 

\section{Prior Knowledge and Outlier Detection}
\label{sec:sec01}

In this project a prior strength test was successfully carried out to determine both the effect of changing the restrictiveness of the prior distribution on the Bayesian fit outcome, as well as the appropriate prior strength to select for the specific simulated strain samples which were later fit using linear regression. The implications on this for industry use are that the extent of the prior distribution's influence on the fit outcome is not subject, but is in the researcher's control. It is possible to run similar tests to assess what is the most appropriate prior standard deviation to use, however due to computational and time demands (approximately 2 hours) this may not be reasonable.\\ 

Outlier detection was also successfully automated for extreme anomalies using Bayesian, resulting in fit outcomes which were sometimes, although often not, more accurate than those resulting from robust (Student's T)  Bayesian regression. This could potentially eliminate much of the subjectivity and tediousness of manual data optimisation that is often a prerequisite to obtaining a sufficiently accurate fit from Least-Squares regression. 

\section{Implications and Limitations of this work.}
\label{sec:sec01}

While simulated strain measurements of varying difficulty were successfully analysed using Bayesian regression in this project,illustrating that Bayesian regression can yield more accurate fit outcomes for samples containing additive noise distributions, it is yet to be proven that Bayesian can be used to successfully analyse a real sample of X-Ray diffraction measurements which suffers from poor grain sampling statistics. Furthermore, the origin of the lack of consistency, and in particular the failure to yield a more accurate stress analysis for samples with errors drawn from one distribution, may not be explained by the technical error described above. 
