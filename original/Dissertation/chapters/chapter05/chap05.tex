%
% File: chap01.tex
% Author: Victor F. Brena-Medina
% Description: Introduction chapter where the biology goes.
%
\let\textcircled=\pgftextcircled
\chapter{Conclusions and Recommendations for Future Work}
\label{chap:intro}

\initial{I}n this project Bayesian regression was utilised to successfully analyse simulated strain measurements with additive noise, proving that, in principle, Bayesian regression can be utilised to aid in the analysis of strain measurements which contain additional uncertainties due to grain sampling.\\

Furthermore, the effect of the prior distribution on the Bayesian fit outcome was examined, finding that numerical changes have a marginal impact on fit outcome, provided the prior is not too restrictive, whereas changes to the likelihood function can yield more accurate, and sometimes bi-modal, fit-outcomes in cases where the measurements contain additive error distributions.\\

The project failed to examine the results of Bayesian fitting on samples with errors drawn from one distribution, presumably due to a technical mistake outlined in section 5.1.1, but did successfully automate the outlier detection and weighting to yield fit outcomes that were of similar quality, if not better, than those obtained from robust Bayesian regression. Importantly, robust Bayesian regression led to fits with more representative confidence intervals than what was (incorrectly) yielded by Least-Squares regression (section 4.4), enabling the observation of bi-modal stress distributions when they arise due to outliers, as well as resulting in a more accurate fitted stress.\\

In order to further this work, it is suggested to attempt to fit real strain measurements which suffer from grain sampling derived uncertainties using the robust Bayesian with outlier detection (as described in section 3.4) to see if the measurements can be analysed without the need for manual optimisation of data. Furthermore, to enable a larger scale of use in the Nuclear industry, PyMC3 methods would benefit from the creation of a user interface for robust Bayesian regression. 

% %A figures matrix.
% \begin{figure}[t!]
% \centering
% \begin{minipage}{3.3cm}
%     \centering
%     \subtop[]{\includegraphics[height=0.28\textheight]{fig01/Nswellings}\label{sf:multiRH02a}}
% \end{minipage}
% \hspace{0.5cm}
% \begin{minipage}{3.3cm}
%     \centering
%     \subtop[]{\includegraphics[height=0.27\textheight]{fig01/Mswellings}\label{sf:multiRH02b}}
% \end{minipage}
% \hspace{1.3cm}
% \begin{minipage}{3.3cm}
%     \centering
%     \subtop[]{\includegraphics[height=0.27\textheight]{fig01/rhd1}\label{sf:multiRH02c}}
% \end{minipage}
% \\ \vspace{0.1cm}
% \begin{minipage}{10cm}
%     \centering
%     \subtop[]{\includegraphics[height=0.145\textheight]{fig01/mutantrhd6}\label{sf:multiRH02d}}
% \end{minipage}
% \\ \vspace{0.1cm}
% \begin{minipage}{10cm}
%     \centering
%     \subtop[]{\includegraphics[height=0.16\textheight]{fig01/auxab}\label{sf:multiRH02e}}
% \end{minipage}
% \mycaption[Hair-forming mutant cells.]{(a) A mutant RH cell. Asterisks show multiple sites of RH initiation in a single root hair cell (indicated by the arrows). Figure reproduced from \cite{rigas01}. (b)~Hair-forming cell with three RH initiation locations. The bar represents $50\mu m$. Figure reproduced from \cite{massuci01}. (c) Large bump in mutant {\itshape rhd1}. Figure reproduced from \cite{griersonRH}. (d) Mutant overexpressing gene {\itshape ROP2}; from right-hand to left-hand, numbers indicate progressive snapshots at different times. RH initiation sites are indicated by the arrows. The bar represents $75\mu m$. Figure reproduced from~\cite{mjones01}. (e)~Mutants affected by auxin. On the left-hand side, RH site is farther away from the apical end (left arrow cap); on the right-hand side, multiple RH locations (arrows). Figure reproduced from~\cite{payne01}.}
% \label{fig:multiRH02}
% \end{figure}

% % A single figure
% \begin{figure}[t!]
% 	\centering
% 	\includegraphics[height=0.35\textheight]{fig01/devepzones}
% 	\mycaption[Developmental zones of an Arabidopsis root.]{Developmental zones of an Arabidopsis root. Figure reproduced from \cite{griersonRH}.}
% 	\label{fig:RHP02}
% \end{figure}

%=========================================================